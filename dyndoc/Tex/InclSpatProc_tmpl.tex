[#preamble+]
\newcommand{\VRE}{V^R}
\newcommand{\WRE}{W^R}
\newcommand{\VR}[1]{\VRE\left(#1\right)}
\newcommand{\WR}[2]{\WRE\left(#1,#2\right)}
\newcommand{\VRI}[2]{\VRE\left(#1|#2\right)}
\newcommand{\VRD}[3][\Lambda]{\VRE_{#1}\left(#2|#3\right)}

\newcommand{\VE}{V}
\newcommand{\WE}{W}
\newcommand{\Par}{\theta}                    % parameter
\newcommand{\ParV}{\Vect{\theta}}            % parameter
\newcommand{\ParVi}[1]{\Vect{\theta}^{(#1)}} % parameter
\newcommand{\Pari}[1]{{\theta}^{(#1)}}
\newcommand{\ParVT}{\Vect{\theta}^\star}     % true parameter
\newcommand{\SpPar}{\Vect{\Theta}}           % parameter space
\newcommand{\Vois}{\mathcal{\Vect{V}}}

\newcommand{\PL}[3]{PL_{#1} \left( #2;#3 \right)}
\newcommand{\LPL}[3]{LPL_{#1} \left( #2;#3 \right)}


\newcommand{\Dom}[1]{\Lambda_{(#1)}}

\newcommand{\V}[1]{\VE\left(#1\right)}
\newcommand{\W}[2]{\WE\left(#1,#2\right)}
\newcommand{\VI}[2]{\VE\left(#1|#2\right)}
\newcommand{\VD}[3][\Lambda]{\VE_{#1}\left(#2|#3\right)}

\newcommand{\VEPar}[1]{V(#1)}
\newcommand{\VPar}[2]{\VE \left(#1 ; #2  \right)}
\newcommand{\VIPar}[3]{\VE \left(#1|#2  ; #3 \right)}

\newcommand{\dVIPar}[4]{\frac{\partial \VE}{\partial \Par_{#1}} \left(#2|#3  ; #4 \right)}
\newcommand{\ddVIPar}[5]{\frac{\partial^2 \VE}{\partial \Par_{#1} \partial \Par_{#2}  } \left(#3|#4  ; #5 \right)}


\newcommand{\dPL}[3]{ \mathbf{PL}_{\Lambda_{#1}}^{(1)} \left( #2 ; #3  \right) }
\newcommand{\dPLt}[2]{ \mathbf{PL}_{\widetilde{\Lambda}}^{(1)} \left( #1 ; #2 \right) }

\newcommand{\dLPL}[3]{ \mathbf{LPL}_{\Lambda_{#1}}^{(1)} \left( #2 ; #3  \right)
 }
\newcommand{\dLPLt}[2]{ \mathbf{LPL}_{\widetilde{\Lambda}}^{(1)} \left( #1 ; #2\right) }

\newcommand{\Estn}[1]{ \widehat{#1}_n }


\newcommand{\SEx}[2]{u_{#1}(#2)}
\newcommand{\SExI}[3]{u_{#1}(#2|#3)}
\newcommand{\VSEx}[1]{\Vect{u}(#1)}                  %% pour les statistiques exhaustives
\newcommand{\VSExI}[2]{\Vect{u} \left(#1 | #2  \right)}

\newcommand{\Dt}{\widetilde{D}}


\newcommand{\MEPar}[3]{W_{#1}\left(#2,#3\right)}
%% \newcommand{\VDPar}[3]{\VEPar{#1}\left(#2|#3\right)} is the same than the previous \VIPar

\newcommand{\Dp}[2]{\left( #1 | #2 \right) }

\newcommand{\intConf}[2][z]{\oint_{#2}^{#1}}
\newcommand{\partFuncE}[2][z]{Z^{#1}_{#2}}
\newcommand{\partFunc}[3][z]{Z^{#1}_{#2}\left(#3\right)}
\newcommand{\corrFuncE}[3][z]{\rho^{#1}_{#2}\left(#3\right)}
\newcommand{\corrFunc}[4][z]{\rho^{#1}_{#2}\left(#3|#4\right)}
\newcommand{\spLE}[2][]{\Pi^{#1}_{#2}}
\newcommand{\spL}[3][]{\Pi^{#1}_{#2}\left(#3\right)}
\newcommand{\spVE}[1][\Lambda]{V_{#1}}
\newcommand{\spV}[2][\Lambda]{V_{#1}\left(#2\right)}
[#end]